\def\thisdir{science/veryhighz/}


\section{Search for Galaxies at $z>7$ with Narrow-Band Imaging
\label{sec:nbf}}

\noindent
\begin{center}
%% Authors
{\bf Ikuru Iwata$^{1}$}\\
$^1$ Subaru Telescope, National Astronomical Observatory of Japan, 650
North Aohoku Place, Hilo, HI 96720, USA
\end{center}
\vspace{0.5cm}

\subsection{Introduction}

Subaru has been one of leading facilities pushing the frontier of
the distant universe. A unique capability of the prime focus camera
(Suprime-Cam) have enabled us to conduct wide-field survey which is
essentially important to find very rare objects such as luminous distant
galaxies. One of the efficient methods to find distant star-forming
galaxies is to detect Ly$\alpha$ emission using narrow-band filter (NBF) 
imaging. A strongly star-forming object with a redshift 
$z = \lambda_\mathrm{NBF} / \lambda_\mathrm{Ly\alpha} -1$ could
appear to be bright compared to those with adjacent broad-band
filters. Galaxies detected with this methods are called as 'Ly$\alpha$
emitters (or LAEs)'. The current most distant galaxy with a spectroscopic 
confirmation is an LAE at $z=7.215$, which was discovered by
\citet{Shibuya2012} using Suprime-Cam with a narrow-band filter NB1006
(central $\lambda$ is 10,052\AA).

Currently a new prime focus camera for Subaru Telescope in optical
wavelength, Hyper Suprime-Cam (HSC), is under testing. HSC has more than
seven times wider field-of-view, and it is expected to enable us
conducting deep surveys much more efficiently than the current
Suprime-Cam. HSC will have a NBF called NB101 which has a central
$\lambda$ is 10,095\AA, which will be used to detect many $z\sim 7.3$
LAEs. 
However, the wavelength of the redshifted Ly$\alpha$ is almost at the
long wavelength limit of the CCDs, and finding galaxies at $z>7.5$ with
cameras with CCDs is impossible. So deep near-IR surveys are mandatory
to push the redshift frontier further.

(Cosmic reionization)


\subsection{Surveys with ULTIMATE-SUBARU}

\par\noindent
[Methods.]

\par\noindent
[What should be clarified with ULTIMATE-SUBARU.]

\subsection{Proposed Observations}

\par\noindent
[Target objects: sample selection, number of objects, number of observing
fields, sky area.]

\par\noindent
[Observing modes: imaging or spectroscopy.]

\par\noindent
[Required observing time:]

\par\noindent
[Special requirements for ULTIMATE-SUBARU other than baseline
specifications, if any.]

\subsection{Synergy and Competitions}

\subsubsection{Synergy with TMT}

\subsubsection{Competitions with other facilities}

Instruments for 8--10m class telescopes.

ELT instruments.

Space-based projects.

\bibliographystyle{apj}
\bibliography{\thisdir nbf}
