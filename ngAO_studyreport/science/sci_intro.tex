\chapter{Science with ULTIMATE-SUBARU 
\label{chap:science}}

In this chapter we introduce science cases proposed to be carried out
with Subaru Telescope Next-Generation AO (ngAO) and clarify
specifications ngAO and associated new instruments should satisfy.

The GLAO system we are studying have following features (see
Chapter~\ref{chap:system_design} for details):

\begin{itemize}
\item Wide-field seeing improvement by correcting WFE caused by ground
      layer of the Earth's atmosphere. Seeing improvement will provide
      not only better angular resolution, but also the significant
      improvement of sensitivity especially for point-like sources. On
      the other hand, the WFE correction (and subsequently angular
      resolution) for individual sources are not as good as classical AO
      systems (such as AO188 of Subaru) which are designed to achieve
      diffraction-limited image for narrow-field. 
\item Number of optical component should be reduced by using the
      Adaptive Secondary Mirror. This means that thermal emission from
      telescope and instrument should be reduced, and that sensitivity
      at longer wavelength ($\gsim 2\mu$m) will be improved.
\end{itemize}

We have developed studies of the science cases under the recognition of
these features, and with strong interactions with technical studies of
the GLAO system and the associated instruments.

We have two primary scientific objectives (or 'Science Drivers') of this
project. The one is `Complete Sensus of Galaxy Evolution with
Large-Scale Near-IR Surveys', and the other is `Discovery of the Most
Distant Galaxies and Understanding of the Process of the Cosmic
Reionization'.

Because GLAO can provide images with better spatial resolution and
improved sensitivity, the GLAO system can be a `significant upgrade
of Subaru Telescope' rather than an introduction of a new
instrument. Various reseaches should be benefitted with the system.
In Septermber 2011 we had a science workshop in the Japanese community
titled `Science Workshop for Subaru Telescope Next-Gen AO System' in
Osaka,
Japan\footnote{\href{http://www.naoj.org/Projects/newdev/ws11b/index.html}{http://www.naoj.org/Projects/newdev/ws11b/}}. 
We also had 'Subaru GLAO Science Workshop' in June 2013 in Hokkaido
Univ. Some Canadian researchers as well as those from Taiwan has
participated the workshop and presented their prospects. 
Through these workshops, we received important suggestions and proposals
for wide-range of researches, such as galaxy evolution, growth of
massive black holes, galaxy archaeology, the Galactic center,
star-forming regions, and exoplanets. From this science workshop we
started more extensive discussions on various science cases, and this
chapter represents the current outcomes of such discussions and studies.
