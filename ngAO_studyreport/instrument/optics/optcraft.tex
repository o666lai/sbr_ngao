\def\thisdir{instrument/optics/}

\begin{center}
\section{Studies of the Optics for the Wide-Field Near-IR Instrument
\label{sec:inst_optics}}
\vspace{0.5cm}

\noindent
\large
%% authors
{\bf T. Yamamuro$^{1}$, K. Motohara$^{2}$, I. Iwata$^{3}$ and Subaru
Telescope Next-Gen AO Instrument sub-working group}\\
$^1$ Optocraft
$^2$ Institute of Astronomy, University of Tokyo
$^3$ Subaru Telescope, National Astronomical Observatory of Japan
\vspace{0.5cm}

\end{center}

以下の検討結果は、オプトクラフト社による
「すばる望遠鏡 次世代広視野補償光学用近赤外線装置 基礎検討」(文書番号
CP0046--11--RP001)からの抜粋である。
なお、報告書には各設計での光学パラメータテーブルも掲載されている。

%%% 20130311 ommitted by Iwata

\subsection{Optical Design without FoV Split}

\subsection{Optical Design with FoV Split}
